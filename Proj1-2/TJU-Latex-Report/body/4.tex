
\section{实验分析}

\subsection{Miss Rate}

如图\ref{fig:mr} 所示:

\begin{figure}[!hbtp]
    \centering
    \subfmacro{ls1}{L1 Cache Size}{.31}
    \subfmacro{a1}{Associativity}{.31}
    \subfmacro{bs1}{Block Size}{.31} 
    \caption{Miss Rate}\label{fig:mr}
\end{figure}

\begin{itemize}
    \item 当 L1 缓存大小增加时,L1MissRate 平均值呈现下降趋势。这是符合预期的,因为更大的缓存大小通常意味着更低的缺失率。
    \item L1 Associativity 值与 L1MissRate 之间的关系似乎并不明确。这可能是因为其他参数的变化对缺失率产生了影响,使得 associativity 的影响不太明显。通常,增加 associativity 可以减少缺失率,但如果其他参数不合适,这种效果可能会受到抵消。
    \item 对于给定的数据,块大小似乎并没有对 L1MissRate 产生显著的影响。这可能是因为块大小的选择需要与其他参数,如缓存大小和 associativity,相匹配才能获得最佳性能。
\end{itemize}


\subsection{AAT}

如图\ref{fig:aatL1} 和 \ref{fig:aatL2} 所示:

\begin{figure}[!hbtp]
    \centering

    \subfmacro{l1_size_assoc_trace1}{Gcc}{.31}
    \subfmacro{l1_size_assoc_trace2}{Go}{.31}
    \subfmacro{l1_size_assoc_trace3}{Perl}{.31}
    \caption{AAT - L1 Cache}\label{fig:aatL1}

\end{figure}

\begin{figure}[!hbtp]
    \centering
    \subfmacro{l2_size_assoc_trace1}{Gcc}{.31}
    \subfmacro{l2_size_assoc_trace2}{Go}{.31}
    \subfmacro{l2_size_assoc_trace3}{Perl}{.31}
    \caption{AAT - L2 Cache}\label{fig:aatL2}

\end{figure}

\begin{itemize}
    \item 增大 L1 Cache Size 或 L1 ASSOC 通常可以降低 AAT。这可能是因为增大这些参数可以增加缓存的容量和关联性,从而降低缺失率和增加缓存命中率。
    \item 与 L1 缓存类似,增大 L2 SIZE 或 L2 ASSOC 也可以降低 AAT。但是,L2 缓存的影响可能不如 L1 缓存明显,因为 L2 缓存通常在 L1 缓存缺失时才被访问。
\end{itemize}

如图 \ref{fig:aatBs} 所示:

\begin{figure}[!htbp]
    \centering
    \subfmacro{blocksize_comparison}{Block Size}{.47}
    \subfmacro{victim_comparison_sorted}{Victim Size}{.47}
    \caption{AAT}\label{fig:aatBs}
\end{figure}


\begin{itemize}
    \item 对于所有三个 trace file,随着 BLOCKSIZE 的增加,AAT 似乎都有所减少,尤其在较小的 BLOCKSIZE 值时下降得较为明显。这可能是因为增大 BLOCKSIZE 可以提高空间局部性,从而增加缓存命中率。但是,BLOCKSIZE 过大可能会导致缓存行里的一些数据没有被利用,从而浪费缓存空间。对于 \verb|go_trace.txt|,在 BLOCKSIZE 较大时,AAT 的减少不如其他两个明显,这可能是因为这个工作负载与其他两个有所不同。
    \item 对于所有三个 trace file,当 Victim Cache SIZE 为 512 时,AAT 值都是最高的。这可能是因为该大小不适合这些特定的工作负载,或者这个大小的 Victim Cache 在逐出数据时可能不够有效。
        随着 Victim Cache SIZE 的增加(从 512 开始),AAT 明显降低。这表明,增大 Victim Cache 的大小可以提高其效率,从而降低 AAT。当 Victim Cache SIZE 达到一定值后,例如 32 或 64,AAT 的下降趋势放缓,表明在这些值之后,进一步增加 Victim Cache 的大小可能不会带来显著的性能提升。
\end{itemize}

\subsection{最优参数组合}

\begin{table}[!htbp]
    \centering
    \caption{最优参数组合}\label{tab:res2}
    \begin{tabular}{cccccc}
        \hline
        Trace & L1 & L2 & blocksize & victim size & Best AAT \\
        \hline
        gcc & 8192, 2 & 65536, 8 & 128 & 4096 & 0.8182 \\
        go & 2048, 2 & 65536, 2 & 128 & 2048 & 0.8466 \\
        perl & 8192, 2 & 65536, 2 & 64 & 4096 & 0.6714 \\
        \hline
    \end{tabular}
\end{table}
